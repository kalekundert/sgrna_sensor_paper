\documentclass[10pt,oneside]{article}
\usepackage{adjustbox}
\usepackage{array}
\usepackage{bbding}
\usepackage[autocite=superscript]{biblatex}
\usepackage{caption}
\usepackage{booktabs}
\usepackage{graphicx}
\usepackage{multicol}
\usepackage{multirow}
\usepackage{seqsplit}
\usepackage{tabularx}
\usepackage{units}

% Delimit paragraphs with newlines rather than spaces.
\setlength{\parindent}{0pt}
\setlength{\parskip}{\baselineskip}

% Use the same font in the main text as in the figures.
\usepackage{fontspec}
\setmainfont{Liberation Sans}

% Double space the draft.
\usepackage{setspace}
\doublespacing

% Tell LaTeX where to find my figures.
\graphicspath{
 {figure_1/}
 {figure_2/}
 {figure_3/}
 {figure_4/}
}

% Tell LaTeX where to find my bibliography.
\addbibresource{curated_refs.bib}

% Don't number sections.
\setcounter{secnumdepth}{0}

% Define a bunch of useful aliases.
\newcommand\fwdrna{ligRNA\textsuperscript{+}}
\newcommand\backrna{ligRNA\textsuperscript{−}}
\newcommand\captitle[1]{\emph{#1}}
\newcommand\seq[1]{\seqsplit{\texttt{#1}}}

\newcommand\latin[1]{\emph{#1}}
\newcommand\etal{\latin{et~al.}}
\newcommand\apo{\latin{apo}}
\newcommand\Apo{\latin{Apo}}
\newcommand\holo{\latin{holo}}
\newcommand\Holo{\latin{Holo}}
\newcommand\invitro{\latin{in vitro}}
\newcommand\Invitro{\latin{In vitro}}
\newcommand\invivo{\latin{in vivo}}
\newcommand\Invivo{\latin{In vivo}}
\newcommand\ecoli{\latin{E.~coli}}
\newcommand\yeast{\latin{S.~cerevisiae}}

\newcommand\evts[1]{\unit[#1]{evt/s}}
\newcommand\hour[1]{\unit[#1]{h}}
\newcommand\g[1]{\unit[#1]{g}}
\newcommand\kV[1]{\unit[#1]{kV}}
\newcommand\liter[1]{\unit[#1]{liter}}
\newcommand\mgmL[1]{\unit[#1]{mg/mL}}
\newcommand\minute[1]{\unit[#1]{min}}
\newcommand\mL[1]{\unit[#1]{mL}}
\newcommand\mM[1]{\unit[#1]{mM}}
\newcommand\molar[1]{\unit[#1]{M}}
\newcommand\ms[1]{\unit[#1]{ms}}
\newcommand\ngx[1]{\unit[#1]{ng}}
\newcommand\nguL[1]{\unit[#1]{ng/μL}}
\newcommand\nM[1]{\unit[#1]{nM}}
\newcommand\pguL[1]{\unit[#1]{pg/μL}}
\newcommand\px[1]{\unit[#1]{px}}
\newcommand\rpm[1]{\unit[#1]{rpm}}
\newcommand\ugmL[1]{\unit[#1]{μg/mL}}
\newcommand\uL[1]{\unit[#1]{μL}}
\newcommand\uM[1]{\unit[#1]{μM}}
\newcommand\UuL[1]{\unit[#1]{U/μL}}
\newcommand\Vcm[1]{\unit[#1]{V/cm}}

\newcommand\namedcite[1]{\citeauthor*{#1}\autocite{#1}}


\newcommand\beginsupplement{%
   \setcounter{table}{0}
   \renewcommand{\thetable}{S\arabic{table}}%
   \setcounter{figure}{0}
   \renewcommand{\thefigure}{S\arabic{figure}}%
}
\newcommand{\fakecaption}{%
  \vskip0.5\baselineskip
  \refstepcounter{table}%
  \tablename\ \thetable%
}

% Commands to define and reference figures and tables.
\newcounter{mainfig}
\newcommand\reffig[1]{%
 Figure \ref{mainfig:#1}%
}
\newcommand\fig[1]{%
 \refstepcounter{mainfig}%
 \label{mainfig:#1}
 \subsection{\reffig{#1}}
}

\newcounter{maintab}
\newcommand\reftab[1]{%
 Table \ref{maintab:#1}%
}
\newcommand\tab[1]{%
 \refstepcounter{maintab}%
 \label{maintab:#1}
 \subsection{\reftab{#1}}
}

\newcounter{suppfig}
\newcommand\refsuppfig[1]{%
 Figure S\ref{suppfig:#1}%
}
\newcommand\suppfig[1]{%
 \refstepcounter{suppfig}%
 \label{suppfig:#1}
 \subsection{\refsuppfig{#1}}
}

\newcounter{supptab}
\newcommand\refsupptab[1]{%
 Table S\ref{supptab:#1}%
}
\newcommand\supptab[1]{%
 \refstepcounter{supptab}%
 \label{supptab:#1}
 \subsection{\refsupptab{#1}}
}

% Commands to format figures in tables in the proper amounts of space.
\newcommand\figonecol[1]{%
 \begin{center}
  \includegraphics[width=3.46in]{#1}%
 \end{center}
}
\newcommand\figtwocol[1]{%
 \makebox[\textwidth][c]{\includegraphics[width=7in]{#1}}%
}
\newcommand\tabonecol[1]{%
 \singlespacing%
 \begin{adjustbox}{width=3.46in,center=\textwidth}%
   \input{#1}%
 \end{adjustbox}%
}
\newcommand\tabtwocol[1]{%
 \singlespacing%
 \begin{adjustbox}{width=7in,center=\textwidth}%
   \input{#1}%
 \end{adjustbox}%
}



\begin{document}

\title{Controlling Cas9 with ligand-activated and ligand-deactivated sgRNAs}
\author{Kale Kundert, James Lucas, Kyle E. Watters, Christina M. Fitzsimmons, 
Benjamin L. Oakes, Christof Fellman, Andrew H. Ng, Ben M. Heineike, Jennifer A.  
Doudna, Hana El-Samad, Tanja Kortemme}
\maketitle{}

\section{Abstract}

\section{Introduction}

\section{Results}

% The results section needs to convince the reader that the central claim is 
% supported by data and logic. Every scientific argument has its own particular 
% logical structure, which dictates the sequence in which its elements should 
% be presented.
%
% Each paragraph in the results section starts with a sentence or two that sets 
% up the question that the paragraph answers. The middle of the paragraph 
% presents data and logic that pertain to the question, and the paragraph ends 
% with a sentence that answers the question.

% Central claim: We can control CRISPRi with ligand-activated and 
% ligand-deactivated sgRNAs.

%¶ Rational design as proof-of-principle and/or to find good insertion sites.

%¶ All three solvent-exposed stems can be used to control the sgRNA.

We began by seeking to understand which of the sgRNA stems would be sensitive 
to the presence of an aptamer.

  %§ We found that designing specific \apo{} and \holo{} state base-pairing 
  %§ interactions was more effective than simply replacing part of a stem with 
  %§ the aptamer.
  %§
  %§  This isn't really a ground-breaking idea...
  
  %§ How to best tie 1e into the argument?  "We convinced ourselves that these 
  %§ designs were truly responding to theophylline by performing titrations as 
  %§ if \reffig{1}e."

%¶ Rational design had much weaker activity \invivo{}.

%¶ Find optimal linker sequences by large-scale FACS screen.  

%¶ 

%¶ 

%¶ 

\section{Discussion}

% The discussion section explains how the results have filled the gap that was 
% identified in the introduction, provides caveats to the interpretation, and 
% describes how the paper advances the field by providing new opportunities. 
% This is typically done by recapitulating the results, discussing the 
% limitations, and then revealing how the central contribution may catalyze 
% future progress.

%¶ Recapitulate results

%¶

\section{Methods}

\subsection{\Invitro{} Cas9 cleavage assay}

Inserting a change.

\Invitro{} transcription: Linear, double-stranded template DNA was acquired 
either by ordering gBlocks® Gene Fragments from IDT (\reffig{1}) or by cloning 
the desired sequence into a pUC vector and digesting it with with EcoRI and 
HindIII (\reffig{2}).  Each construct had the following T7 promoter --- 
\seq{TATAGTAATAATACGACTCACTATAG} --- and a spacer sequence that began with at 
least 3 G's.  The HiScribe™ T7 High Yield RNA Synthesis Kit (NEB E2040S) was 
used to transcribe \ngx{10-50} of DNA template, and the RNA Clean \& 
Concentrator™-25 spin columns (Zymo R1018) were used to remove unincorporated 
nucleotides.

Target DNA: Target DNA was prepared by using inverse PCR to clone the 
appropriate target sequence into the pCR2.1 vector roughly across from its XmnI 
site (i.e.\ in the MCS).  The vector was then digested with XmnI (NEB R0194S) 
as follows: mix \uL{43.5} \nguL{≈500} miniprepped pCR2.1 DNA, \uL{5.0} 10x 
CutSmart buffer, and \uL{1.5} \UuL{20} XmnI; incubate at 37°C until no 
uncleaved plasmid is detectable on a gel (usually \minute{30--60}); dilute to 
\nM{30}; store at -20°C.  

Cas9 reaction: We adapted the following protocol from \namedcite{briner2014}: 
mix \uL{5.0} water or \mM{30} theophylline and \uL{1.5} \uM{1.5} sgRNA; 
incubate at 95°C for \minute{3}, then at 4°C for \minute{1}; add \uL{5.48} water, 
\uL{1.5} 10x Cas9 buffer, and \uL{0.02} \uM{20} Cas9 (NEB M0386T) via master 
mix; incubate at room temperature for \minute{10}; add \uL{1.5} \nM{30} target 
DNA; pipet to mix; incubate at 37°C for \hour{1}; add \uL{0.09} \mgmL{20} RNase~A 
(Sigma R6148), \uL{0.09} \mgmL{20} Proteinase~K (Denville CB3210-5), and 
\uL{2.82} 6x loading dye via master mix; incubate at 37°C for \minute{20}, then 
at 55°C for \minute{20}; run the entire reaction (\uL{18}) on a 1\% 
agarose/TAE/GelRed gel at \Vcm{4.5} for \minute{70}.

Gel quantification: Band intensities were quantified using Fiji (1.51r).  The 
background was subtracted from each image using a \px{50} rolling ball radius.  
The fraction of DNA cleaved in each lane was calculated:

\begin{displaymath}
 \mathrm{f} = \frac{\mathrm{pixels}_\mathrm{2kb}}{\mathrm{pixels}_\mathrm{4kb} 
 + \mathrm{pixels}_\mathrm{2kb}}
\end{displaymath}

The change in cleavage due to ligand was calculated: 

\begin{displaymath}
 \mathrm{Δf} = \mathrm{f}_\mathrm{theo} - \mathrm{f}_\mathrm{apo}
\end{displaymath}

\subsection{CRISPRi assay}

Strain: The strain used for all CRISPRi experiments was \ecoli{} MG1655 with 
dCas9 and ChlorR on a p15A plasmid (pgRNA-bacteria, Addgene 44251), sgRNA and 
AmpR on a pUC plasmid (pdCas9-bacteria, Addgene 44249), and sfGFP 
\autocite{pedelacq2006}, mRFP \autocite{campbell2002}, and KanR chromosomally 
integrated.  This strain was originally described by \namedcite{qi2013}

Flow cytometry: Overnight cultures were started from freshly picked colonies in 
\mL{1} LB with \ugmL{100} carbenicillin and \ugmL{35} chloramphenicol.  The 
next morning, day cultures were started in \mL{15} culture tubes or 24-well 
blocks by inoculating \uL{4} of saturated overnight culture into \mL{1} EZ Rich 
Defined Medium (Teknova M2105) with 0.1\% glucose, \ugmL{1} 
an\-hydro\-tetra\-cycline, \ugmL{100} carbenicillin, \ugmL{35} chloramphenicol, 
and either \mM{1} theophylline or not.  These cultures were then grown for 8h 
at 37°C with shaking at \rpm{225} before GFP and RFP fluorescence were measured 
using a BD LSRII flow cytometer.  Approximately \unit[10,000]{events} were 
recorded for each measurement.  Biological replicates were done on different 
days using different colonies from the same transformation.

Data analysis: Cell distributions were obtained by computing a Gaussian kernel 
density estimation (KDE) over the base-10 logarithms of the measured 
fluorescence values.  The mode was considered to be the center of each 
distribution (e.g.\ for determining fold changes) and was obtained through the 
BFGS maximization of the KDE.

\subsection{FACS screen}

Library cloning: Randomized regions were inserted into the sgRNA using inverse 
PCR with phosphate-modified and HPLC-purified primers containing degenerate 
nucleotides.  The PCR and ligation reactions were setup as follows: mix 
\uL{19.0} water, \uL{2.5} \uM{5} forward primer, \uL{2.5} \uM{5} reverse 
primer, \uL{1.0} \pguL{100} template DNA and \uL{25.0} Q5® High-Fidelity 2x 
Master Mix (NEB M0492L); run PCR according to NEB's recommendations, with a 
\minute{2} extension time and an annealing temperature around 60°C; add \uL{1} 
\UuL{20} DpnI (NEB R0176L); incubate at 37°C for \hour{1}; purify using QIAquick 
spin columns (Qiagen 28106) and elute in \uL{50} water; add \uL{5.67} 10x T4 
ligase buffer and \uL{1} \UuL{400} T4 ligase (NEB M0202L); incubate overnight 
at 16°C; load the ligated DNA onto Zymoclean™ spin columns (Zymo D4002) and 
elute in \uL{10} water to desalt and concentrate it.

Electrotransformation: Electrocompetent cells were prepared as follows: make 
``low-salt'' SOB media: \g{20} bacto-tryptone, \g{5} bacto-yeast extract, 
\mL{2} \molar{5} NaCl, \uL{833.3} \molar{3} KCl, water to 1L, pH to 7.0 with 
NaOH, autoclave 30 min at 121°C; pick a fresh colony and grow overnight in 
\mL{1} SOB; in the morning, inoculate \liter{1} SOB with the entire overnight 
culture; grow at 37°C with shaking at \rpm{225} until OD=0.4 (\hour{≈4}); place 
cells in an ice bath for 10 min to quickly stop their growth; wash with 400 mL 
pre-chilled water, then 200 mL pre-chilled water, then 200 mL pre-chilled 10\% 
glycerol; resuspend in a total volume of \mL{6} pre-chilled 10\% glycerol; make 
\uL{100} aliquots; flash-freeze and store at -80°C.  Electrocompetent cells 
were transformed as follows: thaw competent cells on ice for \minute{10}; pipet 
once to mix cells with \uL{2} \nguL{≈250} library plasmid; shock at \kV{1.8}
with a \ms{5} decay time; immediately add \mL{1} pre-warmed SOC; recover at 
37°C for \hour{1}; dilute into selective liquid media and grow at 37°C with 
shaking at \rpm{225} overnight.  After PCR and ligation, libraries were first 
transformed into electrocompetent Top10 cells, then mira-prepped 
\autocite{pronobis2016}, sequenced, and combined as necessary, then transformed 
again into electrocompetent MG1655 cells already harboring the dCas9 plasmid.

Sorting: Cells were grown as for the CRISPRi assay, but when starting new  
cultures, care was taken to subculture at least 10x more cells than the size of 
the library (often \uL{200}).  Sorting was done using a BD FACSAria II cell 
sorter.  Sorting was no slower than \evts{1000} and no faster than 
\evts{20,000}, with the slower speeds being more accurate and the faster speeds 
being necessary to sort large libraries.  Gates were drawn based on the 
position of the wildtype population if possible, and based on the most extreme 
library members otherwise.  All gates were drawn diagonally in GFP vs.\ RFP 
space.  Sorted cells were collected in \mL{1} SOC and, after sorting, were 
diluted into selective media and grown at 37°C with shaking at \rpm{225} 
overnight.

Screening for \ligrnaB{}: Libraries are as in \refsupptab{libraries}


\section{Figures}

\fig{1}

 \figonecol{figure_1}

 \captitle{Rational design of ligand-sensitive sgRNAs}
 %
 (a) A schematic illustrating our intent to engineer an sgRNA that only adopts 
 its native fold and assembles into a functional Cas9 complex in the presence 
 of a small molecule ligand.
 %
 (b) A map of the sgRNA domains most relevant to this paper 
 \autocite{briner2014}.
 %
 (c) Ligand binding causes the ends of the theophylline aptamer to coalesce 
 into a rigid, stem-like structure \autocite{zimmerman1997}.
 %
 (d) \Invitro{} Cas9 cleavage data for some of the strongest rational designs.  
 Design numbers refer to \refsupptab{rational-designs} and are color-coded by 
 the domain in which the aptamer was inserted.  The percent cleavage values are 
 the average of at least two experiments.  All data shown are from a single 
 gel.
 %
 (e) Representative graded response to increasing concentrations of ligand.

\fig{2}

 \figtwocol{figure_2}

 \captitle{Robust ligRNAs identified by \invivo{} screen}
 %
 (a) A schematic of the screen used to isolate \ligrnaF{}.  The numbers above 
 the arrows indicate approximate library sizes at each step.  G1, G2, R1, and 
 R2 refer to different spacers targeting GFP and RFP, respectively.
 %
 (b,c) Single-cell fluorescence distributions for \ligrnaF{} (teal) and 
 \ligrnaB{} (navy) with (solid lines) and without (dashed lines) theophylline.  
 Control distributions are in grey.  The mode of each distribution is indicated 
 with a plus sign.  Fluorescence values for each cell are normalized by GFP 
 fluorescence for that cell and the modes of the un-repressed control 
 populations measured for that replicate.
 %
 (d) \Invitro{} cleavage data for both ligRNAs in the context of 24 randomly 
 chosen spacers.  The colors indicate the change in the percent of DNA cleaved 
 with and without theophylline.  Each box represents the mean of three 
 measurements with a single spacer.   The grey error bar in the color scale 
 shows the mean and standard deviation for 143 control measurements.
 %
 (e) \Invivo{} theophylline titration for both ligRNAs.  The fluorescence axis 
 is the same as in (b) and (c).
 

\fig{3}

 \figonecol{figure_3}

 \captitle{Parallel control of two genes with two ligands} 
 %
 (a,b) Schematic illustrating the constructs used in the experiment, and the 
 expected consequences of adding theophylline (theo) and 3-methylxanthine (3mx) 
 for each reporter.
 %
 (c) Fluorescence values measured over a \hour{48} timecourse.  Single-cell 
 fluorescence values were measured by flow cytometry.  Bar heights and error 
 bars represent the modes and standard deviations of the cell distributions, 
 respectively.  Unlike in \reffig{2}, fluorescence is normalized by 
 side-scatter (SSC) because both fluorescent channels are being manipulated.

\section{Tables}

\tab{1}

 \tabonecol{table_1/summary_tabular}

 \captitle{Summary of rational design results.}  The results are organized by 
 the design strategy we employed and the domain the sgRNA was inserted into.  
 We tested 97 designs and found 10 that were active, including at least one 
 active design for each domain.  We considered a design active if it exhibited 
 a >15\% change in cleavage in response to theophylline; 15\% is about the 
 point at which the difference is clearly visible on a gel.

\section{Supplementary Materials}

\supptab{rational-designs}

 \tabtwocol{supp_mat/rational_design_sequences}

 \captitle{Complete results from the \invitro{} screen of rational designs.}  
 %
 \#: The number used to refer to a rational design in the main text.
 %
 Strategy: The mechanism by which the design was intended to work.  "Stem 
 Replacement" means a stem in the sgRNA was simply replaced with by the aptamer 
 (possibly with a linker).  "Induced dimerization" means the sgRNA was slit in 
 half, with each half getting part of the aptamer, in the hope that the two 
 halves would dimerize in the presence of ligand.  "Strand displacement" means 
 that we designed strands that could base pair in two ways --- one with wildtype 
 sgRNA 2° structure and the other somehow different --- in the hope that the 
 aptamer would trigger a switch between the two conformations.
 %
 Domain: Where in the sgRNA the aptamer was inserted.  
 %
 Cleavage: The percent of DNA that was cleaved by a design in the \invitro{} 
 assay.  The \apo{} and \holo{} columns refer to the cleavage with and without 
 theophylline, respectively, the Δ column is the difference between those, and 
 the σ column is the standard deviation of the Δ values for the designs with 
 more than one replicate.  All the percentages are rounded to the nearest 1.
 %
 N: The number of replicates for each design.  The reported percentages are the 
 average of each replicate.
 %
 Active: We considered a design to be acitve if it exhibited a >15\% change in 
 cleavage in response to ligand.  15\% is about the point at which the 
 difference is clearly visible on a gel.
 %
 Sequence: The sequence of the design, including the spacer.

\supptab{libraries}

 \tabtwocol{supp_mat/library_sequences/library_tabular}

 \captitle{Library sequences screened via FACS}

\printbibliography[title=References]

\end{document}
